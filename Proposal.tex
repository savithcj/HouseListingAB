\documentclass[letterpaper,12pt]{article}

%PREAMBLE
\usepackage[english]{babel}
\usepackage[margin=2cm]{geometry}
\usepackage{titling}


\usepackage{blindtext}

% Preamble ends here and the document begins
% This block is called the environment
\begin{document}
\title{\Large{\textbf{Database Project Proposal \\ House Listing AB}}}
\author{Oscar Chen, Charles Chukwukaeme, Savith Jayasekera}
\date{February 1, 2019}
% Prints the title, author and date
\setlength{\droptitle}{-2.5cm}
\maketitle


\section{Introduction}

The process of buying and selling a home can be a tedious. This gets more complicated and expensive with the use of realtors who offer services that, although necessary, are not specialized enough to add the value that their fees suggest. The real contribution of a realtor, therefore, ends up been in having exclusive access to the inventory of houses in the market. 
The WelcomeHomeAB project aims to alleviate this problem and open up the home inventory to the public. This website plans to make home buying a breeze by offering an easy to use website where home sellers can list their home. The website will also offer services relating to the sale of a home such as: staging, photos, cleaning, inspection and condo document review.
The motivation behind this project is to make the process of home buying easy and accessible and reduce the associated costs.


\section{Problem Definition}

Typically, real estate properties are bought and sold via networks of realtors in North America. When you put your home up for sale, it is advertised for sale within the real estate industry, licensed agents would make contacts on behalf of the seller and potential buyers. While realtors can help make the transaction and home selling/buying experience relatively smooth, they are paid handsomely for it. 

First of all, realtors are not required legally anywhere during a property transaction. We have been using realtors and their networks to list, buy, and sell real estate properties simply out of tradition. It is perhaps from an antique age where there is no internet, and such human networks were required.

Secondly, in the age of information, home sellers should be able to make their home listing public and online. Similar platforms do exist but not commonly used in Alberta. SaskHouses.com is one such well-established web platform purely serving home sellers and buyers of Saskatchewan. The absence of such established platform in Alberta presents an opportunity in itself. 

Thirdly, with newer generation of people (millennials) becoming a more mature generation with increasing purchase power, their ability to utilize technology for efficiency will make such a digital platform more preferable than using a human realtor agent. The platform should allow sellers and buyers exchange contact information, arrange viewing, etc. The platform should also provide resources and contacts for related services such as home inspection, lawyer service, photography, staging, etc.

Lastly, to be able to save tens of thousands of dollars in commission fees alone is an incentive on its own for buyers and sellers of homes to use such a do-it-yourself sort of platform. There is no standard commission for real estate in Alberta (https://www.reca.ca/consumers/real-estate-101/why-work-with-an-industry-professional/commission-calculations/), typically however, the full commission is 7% of the first $100,000 and 3% on the balance of everything over $100,000 (https://wallerrealestate.ca/how-much-are-realtor-commissions/). For a house sold at $500,000, the commission would equate to $19,000.


\section{Proposed Solution}

The primary goal of our project is to create an online marketing website for private home sales. The users can list their houses for sale, and potential buyers and connect with sellers through these listings. The website will include a commission free model so any DIY seller can list their house and gain the total profit from the sale without having to worry about realtor commissions. In addition to the listings, the website will also be populated with guides on helping sellers to effectively sell their homes. When a seller signs up for the web service, they will also be provided guidelines to obtaining other services usually offered by realtors (for example, contacts to services who provide staging, cleaning or inspection services).
The project will result in one main web service which contains two different “applications”. The first web application will allow users to register to the web service. This application will contain a web form that the users can enter their information and sign up for the service. This application also allows to update their profile, post listings, and provides a portal to communication with potential buyers who are interested in buying their home. The second application allows the admins of the website to update the information about the other “realtor services” provided by the web service.


\section{Motivation}

WelcomeHomeAB plans to accomplish 3 goals: 
•	Demystify and simplify the process of buying and selling a home by hosting a website that will be accessible to the public
•	Simplify the process of buying and selling a home by offering a full suite of services to prepare the home for sale for the home seller and guide the home buyer is making a sound decision when choosing a home
•	Getting rid of the middleman between buyers and sellers and in so doing cut out the traditional exorbitant realtor fees and pass those savings to the customers
On its launch, WelomeHomeAB will be the only website of its kind in the province of Alberta. As homeowning Alberta residents, the developers of this site believe that homebuying can be a stress free and enjoyable process. WelcomeHomeAB will bring the real estate market in Alberta to the technological age.
Other websites of it’s kind in Canada (list some here) offer just a space for real estate listings. WelcomeHomeAB plans to be a one stop gap for homebuyers and sellers offering the full suite of services that is needed from the moment a family decides to list their home to the moment the sale is finalized……MAKE THIS BETTER


\section{Conclusion}

\section{References}

\end{document}
